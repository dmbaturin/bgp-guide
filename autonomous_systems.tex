\chapter{Autonomous systems}

Autonomous system (AS) is defined as a group of connected networks with a single routing policy\cite{rfc1930}.

From technical point of view, an \emph{autonomous system} (AS) is a group of BGP routers that use the same
\emph{autonomous system number} (ASN).

\section{AS numbers}

Autonomous system number is an unsigned 32-bit number, so it can take values from 1 to 4294967294.
Before 2007 16-bit numbers were used but rapid growth of the Internet required range expansion\cite{rfc6793}.
Older BGP implementations may not have 32-bit AS number support, but backwards compatibility mechanisms allow
them to receive routing information from implementations that use 32-bit AS numbers, with some limitations.

Organizational side is more complex. ASN must be unique within its visibility scope to ensure proper
operation of the routing protocol, so there must be a way to ensure uniqueness.

In private internets that do not directly communicate routing information to the Internet the administrator is
responsible for ASN distribution. Technically it is not impossible to use any numbers such as 123,
but it is recommended to use numbers from a special ranges reserved for private use, 64512 to 65535 for
16-bit\cite{rfc1930} and 4200000000 to 4294967294 for 32-bit\cite{rfc6996}.

Two other ranges of AS numbers are reserved for examples and documentation, 64496 to 64511 for 16-bit and
65536 to 65551 for 32-bit\cite{rfc5398}.

Private autonomous systems that use BGP for internal routing do not have to be completely isolated from the Internet,
they only must not include AS numbers from private ranges in routing information sent to Internet peers. Using private
numbers inside the network and globally unique numbers on border routers that advertise network as a whole to the Internet
is a common practice.

Globally unique AS numbers for use in the Internet are assigned by IANA and Regional Internet Registries (RIRs): AfriNIC, APNIC,
ARIN, LACNIC, and RIPE NCC. Registration procedure details depend on the RIR, but generally it requires explanation why
an AS is needed and a yearly registration fee. Typically a company will request an IP address block at the same time.
In case of those autonomous systems registered by RIRs, the routing policy is formally defined and stored in WHOIS database.

\section{Types of autonomous systems}

Autonomous systems can be \emph{single-homed} or \emph{multi-homed}. Multi-homed AS is an AS that has connections to more than
one other AS. Single-homed AS is sometimes referred to as ``stub AS".

Autonomous system is a \emph{transit autonomous system} if it allows traffic from other autonomours systems to pass through it.

\section{Choosing between public and private AS}



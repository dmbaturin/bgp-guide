\chapter{BGP overview}

As we have already learned, BGP is a dynamic routing protocol and its purpose
is to communicate routing information between routers. Routers that run a BGP process
are often referred to as ``BGP speakers''.

Routers that exchange routing information with each other are referred to as ``neighbors''
or ``peers''. BGP does not include any neighbor discovery mechanism and all neighbor relationships
are configured manually. 

BGP neighbors do not simply exchange IP packets, but establish a \term{session}.
The underlying protocol of BGP sessions is TCP, and BGP processes usually listen
for connections on port 179\cite{rfc4271}.

Once a session is established, neighbors start exchanging routing information updates
that may either advertise a newly available route or notify that a previously advertised
route is no longer available (\term{withdraw} a route).

There are two sources of routing information: locally originated routes and routes learned
from other neighbors. Locally originated routes must be configured explicitly. Routes learned
from neighbors are re-advertised to other neighbors by default, unless prohibited by 
routing policy configuration.

Since inter-AS routing decisions are often heavily influenced by cost, corporate relationships,
quality of service, and other parameters that cannot be inferred from routing information itself,
most BGP implementations allow the administrator to configure 

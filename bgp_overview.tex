\chapter{BGP overview}

As we have already learned, BGP is a dynamic routing protocol and its purpose
is to communicate routing information between routers. Routers that run a BGP process
are often referred to as ``BGP speakers''.

Autonomous systems normally have few connection points, so BGP does not include any
neighbor discovery mechanism. To become BGP neighbors, that is, be able to exchange
routing information, routers need to be configured manually. Internally, a BGP session
is a TCP connection, and BGP processes listen on port 179.

Since the main idea of an autonomous system is to hide unnecessary internal structure
details, BGP does not automatically select any routes from the system routing table
for advertising to neighbors either.
The list of networks that will be advertised to neighbors is configured explicitly.
It can be specified manually, or imported from another protocol, e.g. static routes or
OSPF.

Because any details of autonomous system's internals are hidden, all network distances
are expressed in terms of autonomous system numbers. This makes autonomous system number
an important configuration parameter, and BGP implementations require the administrator
to specify it. 

By default, BGP advertises all routes configured in the network list and all routes learned
from neighbors. Advertising a route learned from a neighboring AS means becoming a transit
AS for it, so this is not always desirable. Unconditionally accepting all routes from
neighbors may not be desirable either, for this reason BGP supports filtering both
incoming and outgoing routing updates.

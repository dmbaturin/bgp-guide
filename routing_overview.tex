\chapter{Routing overview}

Network routing is the process of determining packet path.

\section{Network addressing}

Internets consist of hosts and every host is identified by a unique address. Network address is a an unsigned integer number.
Different versions of the IP protocol use different address size, IPv4 addresses are 32 bit, and IPv6 addresses are 128 bit long.

Hosts are organized into networks, for that purpose hosts from the same group are assigned with addresses from a contiguous block
of addresses starting with the same bit sequence. The common part of those addresses is known as \emph{network address}.

It would be extremely inefficient to store information about every single host, so all routing decisions
are taken in terms of networks. When a router receives a packet, it takes its destination address and
checks if it belongs any of the networks the router knows about.

Network and host parts of an address are extracted by applying a \emph{bit mask} which is commonly referred to as 
\emph{subnet mask} in network addressing context.

Suppose we are using $11111111111111111111111100000000_2$ mask (255.255.255.0 in dotted decimal format) that allows for 256 hosts
in a network to extract network and host parts from 192.0.2.44 address ($11000000000000000000001000101100_2$ in binary).

We can use bitwise logical AND operation that keeps bits at positions where the mask bits are set to 1 intact and zeroes the rest.

\begin{tabular}{|l|l|l|}
\hline
AND & 0 & 1 \\
\hline
0 & 0 & 0 \\
\hline
1 & 0 & 1 \\
\hline
\end{tabular}

So we can state that 
\begin{equation}
  network\ address = address\ \mathrm{AND}\ mask
\end{equation}

Different networks have different number of hosts, which requires a way to split the address space into chunks of
different size and a way to determine how many bits of an address identify the network and how many bits
identify the host within that network.

In the early Internet this problem was solved by introduction of \emph{address classes}. The value of the first
bits of the addresses used to define its class, and the class used to define network size.

\begin{tabular}{|l|l|l|l|l|}
\hline
Class & First bits & Size & First network & Last network \\
\hline
A & 0000 & 16,777,216 ($2^{24}$) & 0.0.0.0 & 127.0.0.0 \\
\hline
B & 1000 & 65 536 ($2^{16}$) & 128.0.0.0 & 191.255.0.0 \\
\hline
C & 1100 & 256 ($2^{8}$) & 192.0.0.0 & 223.255.255.0 \\
\hline
D & 1110 & Not defined & Not defined & Not defined \\
\hline
E & 1111 & Not defined & Not defined & Not defined \\
\hline
\end{tabular}

As the Internet grew in size, this approach was proven inefficient. Networks with more than 256 hosts had to
request class B networks that were often too large for them and most of addresses were unused. Every single
network had to be present in the global routing table which inflated routing table size. There was no way
to split a network in parts to optimize internal routing.

\subsection{Classless addressing}

The current network addressing scheme is known as ``classless addressing". Routing approach that is based on it
is called "classless interdomain routing" (CIDR), the addressing scheme is often referred to as	CIDR as well.

A CIDR-formatted address consists of two parts: address and prefix length, separated by ``/", such as 
``2001:db8::1/64" or ``192.0.2.10/24". Prefix length is a decimal number that defines how many bits are used
for the network part of the address.




\section{Routing tables}

Every system (either host or router) keeps a \emph{routing table} that contains \emph{routes} to known prefixes.

A route is a record that may include the following fields:
\begin{itemize}
  \item destination network address (prefix);
  \item next hop address;
  \item outbound interface;
  \item route metric;
  \item metainformation, such as protocol the route is learnt from.
\end{itemize}

An excerpt from a routing table may look like:
\begin{verbatim}
Codes: K - kernel route, C - connected, S - static, R - RIP, O - OSPF,
       I - ISIS, B - BGP, > - selected route, * - FIB route

S>* 0.0.0.0/0 [210/0] via 192.0.2.1, eth0
C>* 192.0.2.0/24 is directly connected, eth0
S>* 203.0.113.0/24 [1/0] via 192.0.2.10, eth0
S>* 203.0.113.128/25 [1/0] via 192.0.2.11, eth0
\end{verbatim}

In the first entry, "S" means the route is statically configured, which is metainformation;
0.0.0.0/0 is the destination prefix; 192.0.2.1 is the next hop address; and eth0 is the
outbound interface.

The routing table is traversed from less specific to more specific routes. Route selection
algorithm follows the "longest match" rule: if more than one route matches the destination,
the most specific one is used.

Suppose the above system receives a packet destined to 203.0.113.250. It matches 0.0.0.0/0
destination (as any other possible address), but there is a route to 203.0.113.0/24, which is
more specific. Furthermore, there is a route to 203.0.113.128.0/25, which is even more specific.
The table does not have any more specific routes, so the route with 203.0.113.128/25 prefix 
will be used.

If the system had a route with e.g. 201.0.113.250/32 prefix in addition to those ones, it would
be used for routing decision for that packet.



\section{Connected routes}


